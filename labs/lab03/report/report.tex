% Options for packages loaded elsewhere
% Options for packages loaded elsewhere
\PassOptionsToPackage{unicode}{hyperref}
\PassOptionsToPackage{hyphens}{url}
%
\documentclass[
  english,
  russian,
  12pt,
  a4paper,
  oneside]{article}
\usepackage{xcolor}
\usepackage[left=3cm,right=1cm,top=2cm,bottom=2cm]{geometry}
\usepackage{amsmath,amssymb}
\setcounter{secnumdepth}{5}
\usepackage{iftex}
\ifPDFTeX
  \usepackage[T1]{fontenc}
  \usepackage[utf8]{inputenc}
  \usepackage{textcomp} % provide euro and other symbols
\else % if luatex or xetex
  \usepackage{unicode-math} % this also loads fontspec
  \defaultfontfeatures{Scale=MatchLowercase}
  \defaultfontfeatures[\rmfamily]{Ligatures=TeX,Scale=1}
\fi
\usepackage{lmodern}
\ifPDFTeX\else
  % xetex/luatex font selection
\fi
% Use upquote if available, for straight quotes in verbatim environments
\IfFileExists{upquote.sty}{\usepackage{upquote}}{}
\IfFileExists{microtype.sty}{% use microtype if available
  \usepackage[]{microtype}
  \UseMicrotypeSet[protrusion]{basicmath} % disable protrusion for tt fonts
}{}
\usepackage{setspace}
% Make \paragraph and \subparagraph free-standing
\makeatletter
\ifx\paragraph\undefined\else
  \let\oldparagraph\paragraph
  \renewcommand{\paragraph}{
    \@ifstar
      \xxxParagraphStar
      \xxxParagraphNoStar
  }
  \newcommand{\xxxParagraphStar}[1]{\oldparagraph*{#1}\mbox{}}
  \newcommand{\xxxParagraphNoStar}[1]{\oldparagraph{#1}\mbox{}}
\fi
\ifx\subparagraph\undefined\else
  \let\oldsubparagraph\subparagraph
  \renewcommand{\subparagraph}{
    \@ifstar
      \xxxSubParagraphStar
      \xxxSubParagraphNoStar
  }
  \newcommand{\xxxSubParagraphStar}[1]{\oldsubparagraph*{#1}\mbox{}}
  \newcommand{\xxxSubParagraphNoStar}[1]{\oldsubparagraph{#1}\mbox{}}
\fi
\makeatother


\usepackage{longtable,booktabs,array}
\usepackage{calc} % for calculating minipage widths
% Correct order of tables after \paragraph or \subparagraph
\usepackage{etoolbox}
\makeatletter
\patchcmd\longtable{\par}{\if@noskipsec\mbox{}\fi\par}{}{}
\makeatother
% Allow footnotes in longtable head/foot
\IfFileExists{footnotehyper.sty}{\usepackage{footnotehyper}}{\usepackage{footnote}}
\makesavenoteenv{longtable}
\usepackage{graphicx}
\makeatletter
\newsavebox\pandoc@box
\newcommand*\pandocbounded[1]{% scales image to fit in text height/width
  \sbox\pandoc@box{#1}%
  \Gscale@div\@tempa{\textheight}{\dimexpr\ht\pandoc@box+\dp\pandoc@box\relax}%
  \Gscale@div\@tempb{\linewidth}{\wd\pandoc@box}%
  \ifdim\@tempb\p@<\@tempa\p@\let\@tempa\@tempb\fi% select the smaller of both
  \ifdim\@tempa\p@<\p@\scalebox{\@tempa}{\usebox\pandoc@box}%
  \else\usebox{\pandoc@box}%
  \fi%
}
% Set default figure placement to htbp
\def\fps@figure{htbp}
\makeatother



\ifLuaTeX
\usepackage[bidi=basic,provide=*]{babel}
\else
\usepackage[bidi=default,provide=*]{babel}
\fi
% get rid of language-specific shorthands (see #6817):
\let\LanguageShortHands\languageshorthands
\def\languageshorthands#1{}


\setlength{\emergencystretch}{3em} % prevent overfull lines

\providecommand{\tightlist}{%
  \setlength{\itemsep}{0pt}\setlength{\parskip}{0pt}}



 
\usepackage[backend=biber,langhook=extras,autolang=other*]{biblatex}
\addbibresource{bib/cite.bib}

\usepackage[]{csquotes}

\usepackage{indentfirst}
\usepackage{float}
\floatplacement{figure}{H}
\usepackage{libertine}
\usepackage{fancyhdr}
\pagestyle{empty}
\renewcommand{\headrulewidth}{0pt}
\makeatletter
\@ifpackageloaded{caption}{}{\usepackage{caption}}
\AtBeginDocument{%
\ifdefined\contentsname
  \renewcommand*\contentsname{Содержание}
\else
  \newcommand\contentsname{Содержание}
\fi
\ifdefined\listfigurename
  \renewcommand*\listfigurename{Список иллюстраций}
\else
  \newcommand\listfigurename{Список иллюстраций}
\fi
\ifdefined\listtablename
  \renewcommand*\listtablename{Список таблиц}
\else
  \newcommand\listtablename{Список таблиц}
\fi
\ifdefined\figurename
  \renewcommand*\figurename{Рисунок}
\else
  \newcommand\figurename{Рисунок}
\fi
\ifdefined\tablename
  \renewcommand*\tablename{Таблица}
\else
  \newcommand\tablename{Таблица}
\fi
}
\@ifpackageloaded{float}{}{\usepackage{float}}
\floatstyle{ruled}
\@ifundefined{c@chapter}{\newfloat{codelisting}{h}{lop}}{\newfloat{codelisting}{h}{lop}[chapter]}
\floatname{codelisting}{Список}
\newcommand*\listoflistings{\listof{codelisting}{Листинги}}
\makeatother
\makeatletter
\makeatother
\makeatletter
\@ifpackageloaded{caption}{}{\usepackage{caption}}
\@ifpackageloaded{subcaption}{}{\usepackage{subcaption}}
\makeatother
\usepackage{bookmark}
\IfFileExists{xurl.sty}{\usepackage{xurl}}{} % add URL line breaks if available
\urlstyle{same}
\hypersetup{
  pdflang={ru-RU},
  hidelinks,
  pdfcreator={LaTeX via pandoc}}


\title{РОССИЙСКИЙ УНИВЕРСИТЕТ ДРУЖБЫ НАРОДОВ\\
\strut \\
Факультет физико-математических и естественных наук\\
\strut \\
Кафедра прикладной информатики и теории вероятностей}
\author{}
\date{}
\begin{document}
\maketitle

\renewcommand*\contentsname{Содержание}
{
\setcounter{tocdepth}{2}
\tableofcontents
}
\listoffigures
\listoftables

\setstretch{1.5}
\newpage

\begin{center}
\vspace*{\fill}

Содержание

1 Цель работы    стр.4

2 Теоретическое введение    стр.5

3 Выполнение лабораторной работы   стр.9

4 Выполнение самостоятельной работы   стр.16

5 Выводы    стр.17

\vspace*{\fill}
\end{center}

\newpage

\begin{center}
\vspace*{\fill}

# Лабораторная работа №3. Язык разметки Markdown

\vspace*{\fill}
\end{center}

\newpage

\subsection{3.1. Цель
работы}\label{ux446ux435ux43bux44c-ux440ux430ux431ux43eux442ux44b}

\emph{Освоение процедуры оформления отчетов с помощью легковесного языка
разметки Markdown}

\newpage

\section{3.2. Теоретическое
введение}\label{ux442ux435ux43eux440ux435ux442ux438ux447ux435ux441ux43aux43eux435-ux432ux432ux435ux434ux435ux43dux438ux435}

\subsection{3.2.1. Базовые сведения о
Markdown}\label{ux431ux430ux437ux43eux432ux44bux435-ux441ux432ux435ux434ux435ux43dux438ux44f-ux43e-markdown}

Чтобы создать заголовок, используйте знак \textbackslash\#, например:

\textbackslash\# This is heading 1

\textbackslash\#\textbackslash\# This is heading 2

\textbackslash\#\textbackslash\#\textbackslash\# This is heading 3

\textbackslash\#\textbackslash\#\textbackslash\#\textbackslash\# This is
heading 4

Чтобы задать для текста полужирное начертание, заключите его в двойные
звездочки:

This text is \textbf{bold}.

Чтобы задать для текста курсивное начертание, заключите его в одинарные
звездочки:

This text is \emph{italic}.

Чтобы задать для текста полужирное и курсивное начертание, заключите его
в тройные звездочки:

This is text is both \textbf{\emph{bold and italic}}.

Блоки цитирования создаются с помощью символа
\textbackslash\textgreater:

\textbackslash\textgreater{} The drought had lasted now for ten million
years, and the reign of the terrible lizards had long since ended. Here
on the Equator, in the continent which would one day be known as Africa,
the battle for existence had reached a new climax of ferocity, and the
victor was not yet in sight. In this barren and desiccated land, only
the small or the swift or the fierce could flourish, or even hope to
survive.

Упорядоченный список можно отформатировать с помощью соответствующих
цифр:

\begin{enumerate}
\def\labelenumi{\arabic{enumi}.}
\tightlist
\item
  First instruction
\item
  Sub-instruction
\item
  Sub-instruction
\item
  Second instruction
\end{enumerate}

Чтобы вложить один список в другой, добавьте отступ для элементов
дочернего списка:

\begin{enumerate}
\def\labelenumi{\arabic{enumi}.}
\tightlist
\item
  First instruction
\item
  Second instruction
\item
  Third instruction
\end{enumerate}

Неупорядоченный (маркированный) список можно отформатировать с помощью
звездочек или тире:

\begin{itemize}
\tightlist
\item
  List item 1
\item
  List item 2
\item
  List item 3
\end{itemize}

Чтобы вложить один список в другой, добавьте отступ для элементов
дочернего списка:

\begin{itemize}
\tightlist
\item
  List item 1
\item
  List item A
\item
  List item B
\item
  List item 2
\end{itemize}

Синтаксис Markdown для встроенной ссылки состоит из части {[}link
text{]}, представляющей текст гиперссылки, и части (file-name.md) --
URL-адреса или имени файла, на который дается ссылка:

\href{file-name.md}{link text}

или

\href{http://example.com/}{link text}

Markdown поддерживает как встраивание фрагментов кода в предложение, так
и их размещение между предложениями в виде отдельных огражденных блоков.
Огражденные блоки кода --- это простой способ выделить синтаксис для
фрагментов кода. Общий формат огражденных блоков кода:

\textbackslash{}\texttt{\textbackslash{}\textbackslash{}}\textbackslash{}\texttt{language\ your\ code\ goes\ in\ here\ \textbackslash{}\textbackslash{}}\textbackslash{}\texttt{\textbackslash{}\textbackslash{}}

\newpage

\subsection{3.2.2. Оформление формул в
Markdown}\label{ux43eux444ux43eux440ux43cux43bux435ux43dux438ux435-ux444ux43eux440ux43cux443ux43b-ux432-markdown}

Внутритекстовые формулы делаются аналогично формулам LaTeX. Например,
формула sin²(𝑥) + cos²(𝑥) = 1 запишется как

\(\\sin^2 (x) + \\cos^2 (x) = 1\)

Выключение формулы:

\[
\\sin^2 (x) + \\cos^2 (x) = 1
\] \{\#eq:eq1\}

Смотри формулу (\autocite*{eq:eq1}).

\subsection{3.2.3. Оформление изображений в
Markdown}\label{ux43eux444ux43eux440ux43cux43bux435ux43dux438ux435-ux438ux437ux43eux431ux440ux430ux436ux435ux43dux438ux439-ux432-markdown}

В Markdown вставить изображение в документ можно с помощью
непосредственного указания адреса изображения. Синтаксис данной команды
выглядит следующим образом:

\begin{figure}

{\centering \includegraphics[width=0.5\linewidth,height=\textheight,keepaspectratio]{./путь/к/изображению.jpg}

}

\caption{Подпись к рисунку}

\end{figure}%

Здесь:

\begin{itemize}
\tightlist
\item
  в квадратных скобках указывается подпись к изображению;
\item
  в круглых скобках указывается URL-адрес или относительный путь
  изображения, а также (необязательно) всплывающая подсказка,
  заключённая в двойные или одиночные кавычки.
\item
  в фигурных скобках указывается идентификатор изображения (\#fig:fig1)
  для ссылки на него по тексту и размер изображения относительно ширины
  страницы (width=90\%)
\end{itemize}

Ссылка на изображение (рис. 3.1) может быть оформлена следующим образом:

(рис. \autocite*{fig:fig1})

\begin{figure}

{\centering \includegraphics[width=0.5\linewidth,height=\textheight,keepaspectratio]{image/MD.png}

}

\caption{Логотип Markdown}

\end{figure}%

\textbf{Рис. 3.1. Подпись к рисунку}

\subsection{3.2.4. Обработка файлов в формате
Markdown}\label{ux43eux431ux440ux430ux431ux43eux442ux43aux430-ux444ux430ux439ux43bux43eux432-ux432-ux444ux43eux440ux43cux430ux442ux435-markdown}

Для компиляции отчетов по лабораторным работам предлагается использовать
следующий Makefile:

\textbackslash{}\texttt{\textbackslash{}\textbackslash{}}\textbackslash{}\texttt{makefile\ all:\ \ \ \ \ @quarto\ render\ clean:\ \ \ \ \ rm\ -rf\ \_output\ cleanall:\ clean\ \ \ \ \ rm\ -rf\ .quarto\ \textbackslash{}\textbackslash{}}\textbackslash{}\texttt{\textbackslash{}\textbackslash{}}


\printbibliography



\end{document}
